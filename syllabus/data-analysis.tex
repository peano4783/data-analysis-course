\documentclass{article}
\usepackage[russian]{babel}
\usepackage{hyperref}
\usepackage[utf8]{inputenc}
\begin{document}
{\LARGE \bf Анализ данных}

\section{Описание курса}
Курс знакомит студентов-программистов с областью Data Science
и специализированным программным обеспечением.

Основная техника обучения данного курса -- метод ситуационного анализа (case studies) с
использованием предварительно подготовленных наборов данных.
Работа с произвольными наборами данных будет рассматриваться в курсе <<вычислительная статистика>>.

\section{Содержание курса}
\begin{enumerate}
\item Ландшафт программного обеспечения для анализа данных
\item Язык Python и специализированные библиотеки
\item Исследовательский анализ и визуализация данных
\item Построение прогностических моделей
\item Обработка больших данных (Apache Spark)
\end{enumerate}


\section{Распределение баллов}

\begin{itemize}
\item Посещаемость и текущая работа в аудитории -- 25 баллов
\item Домашние работы -- 25 баллов
\item Контрольные работы -- 25 баллов
\item Презентация проекта -- 25 баллов
\end{itemize}

Домашние работы сдаются в письменном виде на листочке. Распечатки на принтере не принимаются.

Презентация проекта оценивается максимум в
\begin{itemize}
\item 25 баллов, если презентация успешно проведена в октябре
\item 20 баллов, если презентация успешно проведена в ноябре
\item 15 баллов, если презентация успешно проведена 1--15 декабря
\item 10 баллов, если презентация успешно проведена 16--31 декабря
\end{itemize}
За вопросы по презентациям могут даваться дополнительные баллы.

\section{Кодекс поведения}

Как на лекциях, так и на практике, любой студент имеет право войти в аудиторию и выйти из аудитории в любое время. При этом запрещается спрашивать у преподавателя разрешения выйти или войти (вербально и невербально) и каким-либо иным образом отвлекать преподавателя и других студентов.

Все сказанное в аудитории преподаватель считает услышанным и принятым к сведенью любым студентом, даже если студент в это время физически отсутствовал в аудитории.

Разговоры на {\em лекциях} без разрешения преподавателя не допускаются.
Разговоры на {\em практическом занятии} допускаются только шепотом и только по теме занятия.

Во время {\em контрольной работы} пользование какими-либо источниками информации не допускается, любые электронные технические средства личного пользования должны быть отключены, на поверхности рабочего стола студента должны быть исключительно ручка и бумага, разговоры не допускаются.


\section{Контакт с преподавателем}

По любым вопросам по курсу можно проконсультироваться с преподавателем в его консультационные часы, либо, на усмотрение преподавателя, во время практического занятия. Все общение с преподавателем происходит {\em лично}; преподаватель не отвечает на телефонные звонки, текстовые сообщения, электронные письма, сообщения в социальных сетях, в том числе содержащие слова <<пожалуйста>>, <<срочно>>, <<очень нужно>> и т.п..

Ссылки на учебные материалы публикуются в Twitter преподавателя: \url{https://twitter.com/peano83}.

\end{document}