\documentclass{article}
\usepackage[russian]{babel}
\usepackage{hyperref}
\usepackage[utf8]{inputenc}
\begin{document}

Задание выполняется на отдельном листочке и оценивается в 3 балла.
Для получения баллов за задание необходимо продемонстрировать
свои усилия по решению этого задания (необязательно решать полностью).
Пожалуйста, укажите свою фамилию и имя на листочке.

\medskip

1. Ниже приведен рекурсивный алгоритм $F$. 
\begin{verbatim}
void F(int n) {
  printf("%d\n", n);
  if(n < 5)   {
    F(n+1);
    F(n+3);
  }
}
\end{verbatim}
Какие числа будут напечатаны на экране при выполнении вызова $F(1)$?

\medskip

2. Для набора данных
$$ 2 \; 4\; 3 \; 1 \; 10 \; 34 \; 7 $$
вычислите среднее, медиану, дисперсию и стандартное отклонение.

\medskip

3. Для любого значения $\lambda$ укажите такой $k$, чтобы
$$ \int_0^{+\infty} k e^{-\lambda t} dt = 1.$$
Может ли $\lambda$ быть отрицательным числом? Нулем? Если нет, объясните.

\medskip

4. Сколько способов существует, чтобы разложить 3 письма в 3 конверта? 7 писем в 7 конвертов?
$n$ писем в $n$ конвертов? Ответ обоснуйте.

\medskip

5. Out of 50 students in a certain class, 24 own a bike and 27 own a scooter. If 2 students don't own either of them, what fraction of the students owns both a bike and a scooter? Explain.

\medskip

6. Как называются буквы $\gamma$, $\eta$, $\lambda$, $\nu$, $\sigma$, $\chi$, $\xi$, $\zeta$, $\Pi$, $\Sigma$?


\end{document}