\documentclass[landscape]{slides}
\usepackage[landscape]{geometry}
\usepackage[russian]{babel}
\usepackage{hyperref}
\usepackage[utf8]{inputenc}
\begin{document}
\begin{slide}
Трек Data Science
\begin{itemize}
\item Анализ данных (5 семестр)\\
Вычислительная статистика (6 семестр)
\item Можно было назвать Data Science I и Data Science II.
\item В этом семестре: учимся, работая с <<хорошими>> наборами данных
\\
В следующем семестре: учимся работать с произвольными наборами данных
\end{itemize}
\end{slide}


\begin{slide}
Data Science -- это статистика?
\begin{itemize}
\item ДА: Статистика -- это сбор данных, выявлением природы
и составление прогнозов по этим данным.
<<Data Science>> занимается тем же самым (с применением вычислительной техники).
\item НЕТ: В статистике небольшие выборки ($\leq 100$)
Data Science работает с мегабайтами и гигабайтами данных
=> другие возможности и другие методы.
\item Классическую мат. статистику нужно знать.
\end{itemize}	
\end{slide}


\begin{slide}
Карьера в Data Science
\begin{itemize}
\item Примеры требований: \href{https://krb-sjobs.brassring.com/TGnewUI/Search/home/HomeWithPreLoad?PageType=JobDetails\&partnerid=26059\&siteid=5016\&jobid=67756\&AReq=59032BR\#jobDetails=67756\_5016}{IBM}
\href{https://yandex.ru/jobs/vacancies/dev/res\_dmir/}{Yandex}
\item Требуются как навыки программирования и обработки данных, так и навыки статистики и машинного обучения.
\item Как правило, требуется уровень образования не ниже магистра.
\item Спрос большой, но подходящих кандидиатов мало.
\end{itemize}	
\end{slide}


\begin{slide}
Ландшафт ПО для анализа данных
\begin{itemize}
\item Язык программирования: в принципе, подходит любой, но
предпочтительнее тот, на котором легче выражать идеи -- R, {\bf Python}, Scala, Julia
\item Библиотеки для Python: scikit-learn, matplotlib, numpy, scipy, pandas
\item Big Data: Hadoop, Spark
\item Билиотеки/standalone решения: Vowpal Wabbit, XGBoost, CatBoost
\end{itemize}	
\end{slide}

\begin{slide}
<<Хорошие>> наборы данных
\begin{itemize}
\item scikit-learn: boston, iris, diabetes, digits, linnerud, wine, breast\_cancer
\item Kaggle: \url{https://www.kaggle.com/competitions}\\
Категории: Getting started, Playground, InClass\\
Titanic, House Prices, Digit Recognizer, New York Taxi Trip Duration, etc.
\item R datasets: \href{https://stat.ethz.ch/R-manual/R-devel/library/datasets/html/00Index.html}{package 'datasets'}
\item Используйте эти наборы в своих проектах
\end{itemize}	
\end{slide}

\begin{slide}
Подготовка рабочей среды
\begin{itemize}
\item Установить Python
\item Установить библиотеки scikit-learn, matplotlib, numpy, scipy, pandas
\item Если все установлено правильно, код на следующий слайдах должен исполняться
\end{itemize}	
\end{slide}


\begin{slide}
Загрузка набора данных
\begin{verbatim}
from sklearn.datasets import load_boston
import matplotlib.pyplot as plt

boston = load_boston()
boston['data']
boston['target']
boston.feature_names

plt.scatter([x[1] for x in boston['data']],
  boston['target'])  # smt. crazy, right?
plt.show()
\end{verbatim}
Не очень удобно...
\end{slide}

\begin{slide}
pandas.dataframe
\begin{verbatim}
import pandas as pd 
from sklearn.datasets import load_boston
import matplotlib.pyplot as plt

dataset = load_boston()
df = pd.DataFrame(dataset.data, columns=dataset.feature_names)
df['target'] = dataset.target
df
df.describe()

plt.scatter(df['CRIM'], df['target'])
\end{verbatim}
\end{slide}

\begin{slide}
Домашнее задание
\begin{enumerate}
\item Установите на свой компьютер Python и необходимые библиотеки
\item Загрузите набор данных load\_iris в dataframe.
\item Выведите наименования признаков (feature names)
\item Постройте график зависимости зависимой переменной (target) от каждого признака
\item Напишите (от руки) получившиеся программу и сдайте на следующем занятии
\end{enumerate}	
\end{slide}

\end{document}