\documentclass[landscape]{slides}
\usepackage[landscape]{geometry}
\usepackage[russian]{babel}
\usepackage{hyperref}
\usepackage[utf8]{inputenc}
\usepackage{color}
\begin{document}

\begin{normalsize}


\begin{slide}
\begin{center}
Анализ потоков данных (Часть 2)

Mining Data Streams (Part 2)

\url{http://mmds.org}

Mining of Massive Datasets

Jure Leskovec, Anand Rajaraman, Jeff Ullman
\end{center}
\end{slide}


\begin{slide}
\textbf{\textcolor{blue}{Содержание лекции}}

\begin{itemize}
\item Новые поточные алгоритмы:
  \begin{itemize}
  \item Фильтрация потока данных: фильтры Блума
    \begin{itemize}
    \item Выбор элементов со свойством $x$ из потока
    \end{itemize}
  \item Подсчет различных элементов: алгоритм Флажоле--Мартена
    \begin{itemize}
    \item Количество различных элементов среди последних $k$ элементов потока
    \end{itemize}
  \item Оценка моментов: метод AMS
    \begin{itemize}
    \item Оценка стандартного отклонения последних $k$ элементов
    \end{itemize}
  \item Подсчет частых элементов
  \end{itemize}
\end{itemize}
\end{slide}

\begin{slide}
(1) Фильтрация потоков данных
\end{slide}


\begin{slide}
\textbf{\textcolor{blue}{Фильтрация потоков данных}}

\begin{itemize}
\item Каждый элемент потока данных -- это кортеж (tuple)
\item Для данного набора ключей $S$
\item определить, какие кортежи из потока находятся в $S$
\item Очевидное решение: хеш-таблица
  \begin{itemize}
  \item Но предположим, что памяти для хранения всего $S$ в хеш-таблице недостаточно
    \begin{itemize}
    \item Например, мы можем применять миллионы фильтров на одном и том же потоке
    \end{itemize}
  \end{itemize}
\end{itemize}
\end{slide}


\begin{slide}
\textbf{\textcolor{blue}{Приложения}}

\begin{itemize}
\item Пример: фильтрация спама в электронной почте
  \begin{itemize}
  \item Мы знаем 1 миллиард <<хороших>> электронных адресов
  \item Если сообщение пришло с одного из них, то это НЕ спам
  \end{itemize}
\item Системы publish-subscribe
  \begin{itemize}
  \item Вы собираете много сообщений (новостных статей)
  \item Люди выражают заинтересованность в определенных наборах ключевых слов
  \item Определить, соответствует ли каждое сообщение интересам пользователя
  \end{itemize}
\end{itemize}
\end{slide}



\begin{slide}
\textbf{\textcolor{blue}{Решение--<<пробный дубль>> (1)}}

\begin{itemize}
\item По заданному наборку ключей $S$, которые требуется отфильтровать
\item создать побитовый массив $B$ из $n$ бит, изначально обнулив все элементы
\item Выбрать хеш-функцию $h$ с областью значений $[0,n)$
\item Хешировать каждый элемент $s\in S$ в одну из $n$ урн и положить этот бит равным $1$, т.е. $B[h(s)]=1$
\item Хешировать каждый элемент $a$ потока и выводить только те из них, которые хешируются в бит, равный $1$
  \begin{itemize}
  \item Выводить $a$, если $B[h(a)] == 1$
  \end{itemize}
\end{itemize}
\end{slide}


\begin{slide}
\textbf{\textcolor{blue}{Решение--<<пробный дубль>> (2)}}

Вывести элемент, так как он может быть в $S$.
Элемент хешируется в урну, в которую хешировался
по меньшей мере один элемент из $S$.

Отбросить элемент. Он хешируется в урну с нулевым значением,
поэтому он точно не содержится в $S$.

\begin{itemize}
\item Выдает ложноположительные результаты, но не ложноотрицательные
  \begin{itemize}
  \item Если элемент находится в $S$, то он выводится обязательно, если нет, то он все равно может быть выведен
  \end{itemize}
\end{itemize}
\end{slide}


\begin{slide}
\textbf{\textcolor{blue}{Решение--<<пробный дубль>> (3)}}

\begin{itemize}
\item $|S| = 1$ миллиард адресов электронной почты \\
      $|B| = 1\,GB = 8$ миллиардов бит
\item Если электронный адрес находится в $|S|$, он обязательно хешируется в урну с битом $1$, поэтому он всегда подается на вывод
(нет ложноотрицательных)
\item Около $1/8$ битов положены равными $1$, поэтому около $1/8$
адресов, не принадлежащих $S$, подаются на вывод (ложноположительные)
  \begin{itemize}
  \item В действительности, менее $1/8$, так как в один и тот же бит может хешироваться более одного адреса 
  \end{itemize}
\end{itemize}
\end{slide}


\begin{slide}
\textbf{\textcolor{blue}{Анализ: метание дротиков (1)}}

\begin{itemize}
\item Более точный анализ числа ложноположительных результатов
\item Рассмотрим: Если метать $m$ дротиков в $n$ равновероятных мишений,какова вероятность, что в мишень попадет хотя бы один дротик?
\item В нашем случае:
  \begin{itemize}
  \item Мишени = биты/урны
  \item Дротики = хеш-значения элементов
  \end{itemize}
\end{itemize}
\end{slide}

\begin{slide}
\textbf{\textcolor{blue}{Анализ: метание дротиков (2)}}

\begin{itemize}
\item Дано $m$ дротиков, $n$ мишеней
\item Какова вероятность того, что в мишень попадет хотя бы один дротик?
\end{itemize}
$$ 1 - (1 - 1/n)^m \to 1 - e^{-m/n} $$
Равно $1/e$ при $n\to\infty$\\
Вероятность непопадания дротика в некоторую мишень $X$\\
Вероятность попадания по меньшей мере одного дротика в мишень $X$
\end{slide}

\begin{slide}
\textbf{\textcolor{blue}{Анализ: метание дротиков (3)}}

\begin{itemize}
\item Доля единиц в массиве $B$ = \\
      вероятности ложноположительного результата $=1-e^{-m/n}$
\item Пример: $10^9$ дротиков, $8\cdot 10^9$ мишеней
  \begin{itemize}
  \item Доля единиц в $B$ $=1-e^{-1/8} = 0.1175$
    \begin{itemize}
    \item Сравните с оценкой выше: $1/8 = 0.125$
    \end{itemize}
  \end{itemize}
\end{itemize}
\end{slide}


\begin{slide}
\textbf{\textcolor{blue}{Фильтр Блума}}

\begin{itemize}
\item Рассмотрим: $|S| = m$, $|B|=n$
\item Используем $k$ независимых хеш-функций $h_1,\ldots,h_k$.
\item Инициализация:
  \begin{itemize}
  \item Положить все $B$ равными $0$
  \item Хешировать каждый элемент $s\in S$ с помощью хеш-функции $h_i$,
  положить $B[h_i(s)]=1$ (для каждого $i=1,\ldots,k$)
  \end{itemize}
\item Во время выполнения
  \begin{itemize}
  \item При поступлении элемента потока с ключом $x$
    \begin{itemize}
    \item Если $B[h_i(x)]=1$ {\em для всех} $i=1,\ldots,k$, то считать, что $x$ содержится в $S$
      \begin{itemize}
      \item То есть, $x$ хешируется в урну со значением $1$ для каждой хеш-функции $h_i(x)$
      \end{itemize}
    \item В противном случае, отбросить элемент $x$
    \end{itemize}
  \end{itemize}
\end{itemize}
\end{slide}

\begin{slide}
\textbf{\textcolor{blue}{Фильтр Блума -- Анализ}}

\begin{itemize}
\item Какую долю побитового вектора $B$ составляют единицы?
  \begin{itemize}
  \item Метание $k\cdot m$ дротиков в $n$ мишеней
  \item Поэтому доля единиц $(1-e^{-km/n})$
  \end{itemize}
\item Но у нас $k$ независимых хеш-функций, и элемент $x$ пропускается, только если {\em все} $k$ хешируют элемент $x$ в урну со значением $1$
\item Следовательно, вероятность ложноположительных $=(1-e^{-km/n})^k$.
\end{itemize}
\end{slide}

\begin{slide}
\textbf{\textcolor{blue}{Фильтр Блума -- Анализ (2)}}

\begin{itemize}
\item $m=1$ миллиард, $n=8$ миллиардов
  \begin{itemize}
  \item $k=1$: $(1-e^{-1/8}) = 0.1175$
  \item $k=2$: $(1-e^{-1/4})^2 = 0.0493$
  \end{itemize}
\item Что происходит при увеличении $k$?
  \begin{itemize}
  \item <<Оптимальное>> значение $k$: $n/m\ln(2)$
    \begin{itemize}
    \item В нашем случае: Оптимальное $k=8\ln(2) = 5.54\approx 6$
      \begin{itemize}
      \item Погрешность при $k=6$: $(1-e^{-1/6})^2 = 0.0235$
      \end{itemize}
    \end{itemize}
  \end{itemize}
\end{itemize}
\end{slide}

\begin{slide}
\textbf{\textcolor{blue}{Фильтр Блума -- Резюме}}

\begin{itemize}
\item Фильтры Блума гарантируют отсутствие ложноотрицательных и требуют ограниченный объем памяти
  \begin{itemize}
  \item Подходит для предварительной обработки до более дорогостоящих проверок
  \end{itemize}
\item Подходит для реализации на аппаратном обеспечении
  \begin{itemize}
  \item Вычисление хеш-функций может быть распараллелено
  \end{itemize}
\item Что лучше: $1$ большой $B$ или $k$ малых $B$?
  \begin{itemize}
  \item Одинаково: $(1-e^{-km/n})^k$ против $(1-e^{-m/(n/k)})^k$
  \item Но хранение одного большого $B$ проще
  \end{itemize}
\end{itemize}
\end{slide}

\begin{slide}
Подсчет различных элементов
\end{slide}


\begin{slide}
\textbf{\textcolor{blue}{Подсчет различных элементов}}

\begin{itemize}
\item Задача:
  \begin{itemize}
  \item Поток данных состоит из универсума элементов, выбранных из множества размера $N$
  \item Сохранять счетчик количества различных элементов, встречавшихся до этого момента
  \end{itemize}
\item Очевидный подход:
  Хранить множество встречавшихся элементов
  \begin{itemize}
  \item То есть, хранить хеш-таблицу всех различных встречавшихся элементов
  \end{itemize}
\end{itemize}
\end{slide}

\begin{slide}
\textbf{\textcolor{blue}{Приложения}}

\begin{itemize}
\item Сколько разных слов найдено в веб-страницах на сайте?
  \begin{itemize}
  \item Необычайно малое или большое количество слов может означать искусственность страниц (спам?)
  \end{itemize}
\item Сколько различных веб-страниц клиент запрашивает в течение недели?
\item Сколько различных продуктов продано в последнюю неделю?
\end{itemize}
\end{slide}


\begin{slide}
\textbf{\textcolor{blue}{Использование малого хранилища}}

\begin{itemize}
\item Настоящая задача: Что, если у нас нет места для хранения набора элементов, которые нам встречались?
\item Оценить количество несмещенным образом
\item Принять то, что в подсчете может быть небольшая погрешность, но поставить ограничение на вероятность того, что ошибка слишком большая
\end{itemize}
\end{slide}


\begin{slide}
\textbf{\textcolor{blue}{Подход Флажоле--Мартена}}

\begin{itemize}
\item Выбрать хеш-функцию $h$, отображающую каждый из $N$ элементов в не менее чем $\log_2 N$ бит
\item Для каждого элемента потока $a$ пусть $r(a)$ -- количество хвостовых нулей в $h(a)$
  \begin{itemize}
  \item $r(a) = $ позиция первой единицы, считая справа
    \begin{itemize}
    \item Например, если $h(a)=12$, то $12$ -- это $1100$ в двоичной системе, поэтмоу $r(a)=2$.
    \end{itemize}
  \end{itemize}
\item Записать $R=$ максимальный встреченный $r(a)$
  \begin{itemize}
  \item $R = \max_a r(a)$ по всем пройденным элементам $a$
  \end{itemize}
\item Оценка количества различных элементов $=2^R$
\end{itemize}

\end{slide}


\begin{slide}
\textbf{\textcolor{blue}{Почему это работает: интуитивно}}

\begin{itemize}
\item Очень-очень приблизительная прикидка, почему алгоритм Флажоле--Мартена работает:
  \begin{itemize}
  \item $h(a)$ хэширует $a$ с равной вероятностью в любую из $N$ значений
  \item Поэтому $h(a)$ -- последовательность $\log_2 N$ бит, в которой
$2^{-r}$ доля всех $a$ имеют хвост длиной $r$ нулей
    \begin{itemize}
    \item Около 50\% $a$ хешируются в $***0$
    \item Около 25\% $a$ хешируются в $**00$
    \item Таким образом, если самый длинный встретившийся хвост равен $r=2$ (т.е., окончания хеша элемента $*100$), то, вероятно, мы встретили около 4 различных элементов
    \end{itemize}
  \item Следовательно, требуется хешировать $2^r$ элементов до того, как нам встретится таковой с нулевым суффиксом длины $r$
  \end{itemize}
\end{itemize}
\end{slide}


\begin{slide}
\textbf{\textcolor{blue}{Почему это работает: более формально}}

\begin{itemize}
\item Теперь покажем, почему алгоритм Флажоле--Мартена работает
\item С формальной точки зрения покажем, что вероятность нахождения хвоста из $r$ нулей
  \begin{itemize}
  \item Стремится к $1$ при $m>>2^r$
  \item Стремится к $0$ при $m<<2^r$
  \end{itemize}
  где $m$ -- количество различных элементов, встретившихся в потоке
\item Таким образом, $2^R$ почти всегда будет около $m$!
\end{itemize}
\end{slide}


\begin{slide}
\textbf{\textcolor{blue}{Почему это работает: более формально}}

\begin{itemize}
\item Какова вероятность того, что данная $h(a)$ заканчивается на не менее чем $r$ нулей? -- $2^{-r}$
  \begin{itemize}
  \item $h(a)$ хеширует элементы равномерно случайным огбразом
  \item Вероятность того, что случайное число оканчивается на не менее чем $r$ нулей -- $2^{-r}$
  \end{itemize}
\item Тогда вероятность НЕ встретить хвост длины $r$ среди $m$ элементов:
$$ (1-2^{-r})^m $$
\end{itemize}
\end{slide}


\begin{slide}
\textbf{\textcolor{blue}{Почему это работает: более формально}}

\begin{itemize}
\item Заметим: $(1-2^{-r})^m = (1-2^{-r})^{2^r(m2^{-r})}\approx e^{-m2^{-r}}$
\item Вероятность НЕ встретить хвост длины $r$:
  \begin{itemize}
  \item Если $m<<2^r$, то вероятсность стремится к $1$
    \begin{itemize}
    \item $(1-2^{-r})^m\approx e^{-m2^{-r}} = 1$ при $m/2^r\to 0$
    \item Следовательно, вероятность встретить хвост длины $r$ стремится к $0$
    \end{itemize}
  \item Если $m>>2^r$, то вероятность стремится к $0$
    \begin{itemize}
    \item $(1-2^{-r})^m \approx e^{-m2^{-r}}=0$ при $m/2^r\to\infty$
    \item Следовательно, вероятность встретить хвост длины $r$ стремится к $1$
    \end{itemize}
  \end{itemize}
\item Таким образом, $2^R$ будет почти всегда около $m$!
\end{itemize}
\end{slide}


\begin{slide}
\textbf{\textcolor{blue}{Почему это не работает}}

\begin{itemize}
\item $E[2^R]$ в действительности бесконечно
  \begin{itemize}
  \item Вероятность уменьшается в два раза, когда $R\to R+1$, но значение удваивается
  \end{itemize}
\item Обходные пути задействуют использование множества хеш-функций $h_i$ и получения множества выборок $R_i$
\item Как выборки $R_i$ комбинируются?
  \begin{itemize}
  \item Среднее? Что, если одно очень большое значение $2^{R_i}$?
  \item Медиана? Все оценки -- степени числа $2$
  \item Решение
    \begin{itemize} 
    \item Разбить ваши выборки на малые группы
    \item Взять медиану групп
    \item Затем взять среднее медиан
    \end{itemize}
  \end{itemize}
\end{itemize}
\end{slide}



\begin{slide}
(3) Вычисление моментов
\end{slide}


\begin{slide}
\textbf{\textcolor{blue}{Обобщение: моменты}}

\begin{itemize}
\item Пусть в потоке есть элементы, выбираемые из множества $A$ из $N$ значений
\item Пусть $m_i$ -- число раз, когда значение $i$ встречается в потоке
\item $k$-й момент -- это
$$ \sum_{i\in A} (m_i)^k $$
\end{itemize}
\end{slide}


\begin{slide}
\textbf{\textcolor{blue}{Частные случаи}}
$$ \sum_{i\in A} (m_i)^k $$

\begin{itemize}
\item $0$-й момент = число различных элементов
  \begin{itemize}
  \item Только что рассмотренная задача
  \end{itemize}
\item $1$-й момент = count of the numbers of elements = длина потока
  \begin{itemize}
  \item Легко вычислить
  \end{itemize}
\item $2$-й момент = surprise number $S$ =
мера того, насколько распределение неравномерно
\end{itemize}
\end{slide}


\begin{slide}
\textbf{\textcolor{blue}{Пример: Surprise Number}}

\begin{itemize}
\item Поток длины $100$
\item $11$ различных значений
\item Item counts: $10$, $9$, $9$, $9$, $9$, $9$, $9$, $9$, $9$, $9$, $9$ \\
      Surprise $S=910$
\item Item counts: $90$, $1$, $1$, $1$, $1$, $1$, $1$, $1$, $1$, $1$, $1$ \\
      Surprise $S=8110$
\end{itemize}
\end{slide}


\begin{slide}
\textbf{\textcolor{blue}{Метод AMS}}

\begin{itemize}
\item Метод AMS годится для всех моментов
\item Дает несмещенную оценку
\item Мы сосредоточимся на втором моменте $S$
\item Мы выбираем и следим за многими величинами $X$:
  \begin{itemize}
  \item Для каждой величины $X$ сохраняем $X.el$ и $X.val$
    \begin{itemize}
    \item $X.el$ соответствует элементу $i$
    \item $X.val$ соответсвует счетчику элемента $i$
    \end{itemize}
  \item Заметим, что это требует, чтобы счетчик был в оперативной памяти,
  поэтому количество $X$'ов ограничено
  \end{itemize}
\item Наша цель -- вычислить $S=\sum_i m_i^2$
\end{itemize}
\end{slide}


\begin{slide}
\textbf{\textcolor{blue}{Одна случайная величина ($X$)}}

\begin{itemize}
\item Чему положить $X.val$ и $X.el$?
  \begin{itemize}
  \item Пусть поток имеет длину $n$ (ослабим это условие позже)
  \item Выбрать случайный момент времени $t$ ($t<n$) для начала, так чтобы
  любой момент времени был одинаково вероятен
  \item Пусть в момент времени $t$ поток имеет элемент $i$. Положить $X.el=i$.
  \item Затем мы храним счетчик $c$ ($X.val=c$) числа значений $i$
  в потоке начиная с выбранного момента времени $t$
  \end{itemize}
\item Тогда оценка второго момента ($\sum_i m_i^2$) равна:
$$ S = f(X) = n(2\cdot c-1) $$
  \begin{itemize}
  \item Отметим, что мы храним множество $X$'ов $(X_1,X_2,\ldots,X_k)$
  и наша окончательная оценка будет $S=1/k \sum_j^k f(X_j)$
  \end{itemize}
\end{itemize}
\end{slide}


\begin{slide}
\textbf{\textcolor{blue}{Анализ мат. ожидания}}

\begin{itemize}
\item Второй момент $S=\sum_i m_i^2$
\item $c_t,\ldots$ -- сколько раз элемент в момент времени $t$ появляется начиная с момента времени $t$ ($c_1=m_a$, $c_2=m_a-1$, $c_3=m_b$)
\item $E[f(X)] = \frac 1n \sum_{t=1}^n n(2c_t-1)$
$ = \frac 1n \sum_i n(1+3+5+\ldots+2m_i-1) $
\end{itemize}
\end{slide}


\begin{slide}
\textbf{\textcolor{blue}{Анализ мат. ожидания}}

\begin{itemize}
\item $E[f(X)] = \frac 1n \sum_i n(1+3+5+\ldots+2m_i-1) $
  \begin{itemize}
  \item Небольшое стороннее вычисление: $(1+3+5+\ldots+2m_i-1) =$
$$ \sum_{i=1}^{m_i}(2i-1) = 2 \frac {m_i(m_i+1)}2 - m_i = (m_i)^2 $$
  \end{itemize}
\item Тогда $E[f(X)] = \frac 1n \sum_i n (m_i)^2$
\item Следовательно, $E[f(X)] = \sum_i (m_i)^2 = S$
\item Получаем второй момент (в смысле мат. ожидания)!
\end{itemize}
\end{slide}


\begin{slide}
\textbf{\textcolor{blue}{Моменты высших порядков}}

\begin{itemize}
\item Для оценки $k$-го момента используем по существу тот же алгоритм, но заменяем оценку:
  \begin{itemize}
  \item Для $k=2$ было $n(2\cdot c-1)$
  \item Для $k=3$ используем: $n(3\cdot c^2-3c+1)$ (где $c=X.val$)
  \end{itemize}
\item Почему?
  \begin{itemize}
  \item Для $k=2$: Вспомним, что у нас было $(1+3+5+\ldots+2m_i-1)$ и мы показали, что члены
$c2-1$ (для $c=1,\ldots,m$) в сумме дают $m^2$
    \begin{itemize}
    \item $\sum_{c=1}^m 2c - 1 = \sum_{c=1}^m c^2 - \sum_{c=1}^m(c-1)^2=m^2$
    \item Следовательно: $2c-1=c^2 - (c-1)^2$
    \end{itemize}
  \item Для $k=3$: $c^3-(c-1)^3 = 3c^2-3c+1$
  \end{itemize}
\item В общем случае: Оценка $=n(c^k-(c-1)^k)$
\end{itemize}
\end{slide}


\begin{slide}
\textbf{\textcolor{blue}{Комбинация выборок}}

\begin{itemize}
\item На практике:
  \begin{itemize}
  \item Вычислить $f(X) = n(2c-1)$ для стольких переменных $X$, сколько поместится в памяти
  \item Усреднить их по группам
  \item Взять медиану средних
  \end{itemize}
\item Задача: Потоки бесконечны
  \begin{itemize}
  \item Предполагалось, что существует число $n$, число позиций в потоке
  \item Но реальные потоки продолжаются бесконечно, так что $n$ -- это переменная -- число полученных входных данных
  \end{itemize}
\end{itemize}
\end{slide}


\begin{slide}
\textbf{\textcolor{blue}{Потоки бесконечны: корректировки}}

\begin{itemize}
\item (1) Величины $X$ имеют $n$ в качестве множителя -- хранить $n$ отдельно; просто храним счетчик в $X$
\item (2) Пусть можно хранить только $k$ счетчиков.\\
Мы должны выбрасывать некоторые $X$ с течением времени:
  \begin{itemize}
  \item Цель: Каждое начальное время $t$ выбирается с вероятностью $k/n$
  \item Решение: (выборка фиксированного объема!)
    \begin{itemize}
    \item Выбрать первые $k$ моментов времени для $k$ величин
    \item При поступлении $n$-го элемента ($n>k$), включить его в выборку с вероятностью $k/n$
    \item Если элемент включен в выборку, выбросить одну из хранящихся величин $X$, с равной вероятностью
    \end{itemize}
  \end{itemize}
\end{itemize}
\end{slide}


\begin{slide}
Подсчет наборов элементов
\end{slide}



\begin{slide}
\textbf{\textcolor{blue}{Подсчет наборов элементов}}

\begin{itemize}
\item Новая задача: Дан поток; какие элементы появляются чаще, чем $s$ раз в окне?
\item Возможное решение: Представлять себе поток урн как один бинарный поток на элемент
  \begin{itemize}
  \item $1$ = элемент есть; $0$ = элемента нет
  \item Использовать DGIM, чтобы оценить количество единиц во всех элементах
  \end{itemize}
\end{itemize}
\end{slide}


\begin{slide}
\textbf{\textcolor{blue}{Обобщения}}

\begin{itemize}
\item В принципе, можно подсчитать частые пары или более большие наборы одинаковым образом
  \begin{itemize}
  \item Один поток на набор элементов
  \end{itemize}
\item Недостатки:
  \begin{itemize}
  \item Только приблизительно
  \item Количество наборов элементов слишком велико
  \end{itemize}
\end{itemize}
\end{slide}


\begin{slide}
\textbf{\textcolor{blue}{Экспоненциально затухающие окна}}

\begin{itemize}
\item Экспоненциально затухающие окна: Эвристика для выбора вероятных частых (наборов) элементов
  \begin{itemize}
  \item Какие наиболее популярные фильмы <<в текущий момент>>
    \begin{itemize}
    \item Вместо вычисления количества среди последних $N$ элементов <<в лоб>>
    \item Вычислить гладкое объединение по всему потоку
    \end{itemize}
  \end{itemize}
\item Если поток -- это $a_1,a_2,\ldots$, и мы берем сумму потока, взять за ответ в момент времени $t$:
$$ \sum_{i=1}^t a_i(1-c)^{t-i} $$
  \begin{itemize}
  \item $c$ -- константа, предположительно очень малая, порядка $10^{-6}$ или $10^{-9}$
  \end{itemize}
\item При поступлении нового $a_{t+1}$:\\
Умножить текущую сумму на $(1-c)$ и добавить $a_{t+1}$
\end{itemize}
\end{slide}


\begin{slide}
\textbf{\textcolor{blue}{Пример: подсчет элементов}}

\begin{itemize}
\item Если каждый $a_i$ -- <<элемент>>, можно вычислить характеристическую функцию
каждого возможного элемента $x$ в виде экспоненциально затухающего окна
  \begin{itemize}
  \item То есть: $\sum_{i=1}^t \delta_i\cdot(1-c)^{t-i}$\\
  где $\delta_i=1$, если $a_i=x$, и $0$ в противном случае
  \item Предположим, что для каждого элемента $x$ у нас есть бинарный поток
  ($1$, если $x$ встречается, $0$, если $x$ не встречается)
  \item Поступает новый элемент $x$:
    \begin{itemize}
    \item Умножить все счетчики на $(1-c)$
    \item Добавить $+1$ к счетчику для элемента $x$
    \end{itemize}
  \end{itemize}
\item Назовем эту сумму <<весом>> элемента $x$
\end{itemize}
\end{slide}


\begin{slide}
\textbf{\textcolor{blue}{Скользящие и затухающие окна}}

\begin{itemize}
\item Важное свойство: Сумма по всем весам\\
$ \sum_t(1-c)^t $ равна $1/[1-(1-c)]=1/c$
\end{itemize}
\end{slide}


\begin{slide}
\textbf{\textcolor{blue}{Пример: подсчет элементов}}

\begin{itemize}
\item Какие фильмы наиболее популярны <<в данный момент>>?
\item Пусть хотим найти фильмы с весом $>\frac 12$
  \begin{itemize}
  \item Важное свойство: Сумма по всем весам\\
  $\sum_t(1-c)^t$ равна $1/[1-(1-c)]=1/c$
  \end{itemize}
\item Таким образом:
  \begin{itemize}
  \item Не может существовать более $2/c$ фильмов с весом $\frac 12$ или больше
  \end{itemize}
\item Следовательно, $2/c$ -- граница числа фильмов, подсчитываемых в любой момент времени
\end{itemize}
\end{slide}

\begin{slide}
\textbf{\textcolor{blue}{Обобщение на набор элементов}}

\begin{itemize}
\item Посчитать (некоторые) наборы элементов в E.D.W.
  \begin{itemize}
  \item Какие наборы элементы наиболее популярны в настоящий момент?
    \begin{itemize}
    \item Проблема: Слишком много наборов элементов для того, чтобы хранить счетчики их всех в памяти
    \end{itemize}
  \end{itemize}
\item Когда поступает урна $B$:
  \begin{itemize}
  \item Умножить все счетчики на $(1-c)$
  \item Для неподсчитанных элементов $B$ создать новый счетчик
  \item Добавить $1$ к счетчику любого элемента $B$ и любого набора элементов, содержащихся в $B$, которые уже подсчитываются
  \item Удалить счетчики $<\frac 12$
  \item Инициализировать новые счетчики (следующий слайд)
  \end{itemize}
\end{itemize}
\end{slide}


\begin{slide}
\textbf{\textcolor{blue}{Инициализация новых счетчиков}}

\begin{itemize}
\item Создать счетчик для набора множеств $S\subseteq B$, если каждое собственное подмножество $S$ имело счетчик до поступления в урну $B$
  \begin{itemize}
  \item Интуитивно: Если все подмножества $S$ подсчитываются, это значит, что он <<частые/популярные>>, и поэтому у $S$ есть потенциал быть <<популярным>>
  \end{itemize}
\item Пример:
  \begin{itemize}
  \item Начинаем подсчитывать $S=\{i,j\}$ ттогда $i$ и $j$ подсчитывались до того, как мы встретили $B$
  \item Начинаем подсчитывать $S=\{i,j,k\}$ ттогда $\{i,j\}$, $\{i,k\}$ и $\{j,k\}$ все подсчитывались того, как мы встретили $B$
  \end{itemize}
\end{itemize}
\end{slide}


\begin{slide}
\textbf{\textcolor{blue}{Сколько счетчиков нужно?}}

\begin{itemize}
\item Счетчики для единственных элементов $<(2/c)\cdot$среднее количество элементов в урне
\item Счетчики для более больших наборов элементов = ??
\item Но мы остаемся консервативными о начальных счетчиках больших наборов
  \begin{itemize}
  \item Если бы мы подсчитывали каждый набор, который нам встретился, одна урна из 20 элементов инициализировала бы 1 M счетчиков
  \end{itemize}
\end{itemize}
\end{slide}



\end{normalsize}

\end{document}